\documentclass[10pt]{article}
\usepackage[margin=1in]{geometry}
\usepackage{amsmath, amssymb}
\usepackage{graphicx}
\usepackage{booktabs}
\usepackage{caption}
\usepackage{hyperref}
\usepackage{enumitem}
\usepackage{fancyhdr}

\pagestyle{fancy}
\fancyhf{}
\lhead{ECEN 4293 -- Numerical Methods with Python}
\rhead{Timing Report Analysis Template}
\rfoot{\thepage}

\begin{document}

\begin{center}
{\Large\bfseries Timing Report Analysis and Regression Summary}\\[0.5em]
{\large Oklahoma State University -- School of Electrical and Computer Engineering}\\[0.3em]
Instructor: James E. Stine and Marcus Mellor \\[0.5em]
{\small This report template should accompany your Python output and figures.}
\end{center}

\hrule
\vspace{1em}

\section*{1.\quad Report and Path Context}

\begin{itemize}
  \item \textbf{Report file:} \texttt{<filename>} \hfill \textbf{Date:} \texttt{<date>}
  \item \textbf{Operating corner:} \texttt{<corner>} \hfill \textbf{Library:} \texttt{<libname>}
  \item \textbf{Clock period:} \texttt{<value>}~ns
  \item \textbf{Path:} \texttt{<startpoint>} $\rightarrow$ \texttt{<endpoint>} \quad
        \textbf{Group:} \texttt{<group>} \quad
        \textbf{Type:} \texttt{max/min}
  \item \textbf{Arrival:} \texttt{<value>}~ns \quad
        \textbf{Required:} \texttt{<value>}~ns \quad
        \textbf{Slack:} \texttt{<value>}~ns
  \item \textbf{Stages:} \texttt{<count>} \quad
        \textbf{Pairing rule:} cell \texttt{incr} $\leftrightarrow$ preceding net (Cap, Fanout)
  \item \textbf{Units:} Capacitance = \texttt{<unit>}, Slew = \texttt{<unit>}, Delay = \texttt{ns}, Fanout = \texttt{count}
\end{itemize}

\section*{2.\quad Data Summary}

We extracted $N=\langle N\rangle$ cell–net pairs with nonzero incremental delay.  
Missing fields handling: \texttt{<describe how missing fanout/cap/slew were treated>}.

\begin{table}[h!]
\centering
\caption{Summary Statistics of Extracted Stage Data}
\begin{tabular}{lrrrr}
\toprule
Variable & Mean & Std Dev & Min & Max \\
\midrule
Capacitance & \texttt{<..>} & \texttt{<..>} & \texttt{<..>} & \texttt{<..>} \\
Fanout & \texttt{<..>} & \texttt{<..>} & \texttt{<..>} & \texttt{<..>} \\
Input Slew & \texttt{<..>} & \texttt{<..>} & \texttt{<..>} & \texttt{<..>} \\
Incremental Delay & \texttt{<..>} & \texttt{<..>} & \texttt{<..>} & \texttt{<..>} \\
\bottomrule
\end{tabular}
\end{table}

\section*{3.\quad Regression Models and Fit Quality}

\subsection*{(a) Univariate Model}
\[
t_{\text{pd}} = a + b\,C
\]
\begin{itemize}
  \item $a = \langle a \rangle$~ns (intrinsic delay)
  \item $b = \langle b \rangle$~ns/unit-Cap (sensitivity)
  \item $R^2 = \langle R^2_1 \rangle$
  \item RMSE = \langle RMSE \rangle~ns
\end{itemize}
Interpretation: Every additional unit of capacitance increases incremental delay by approximately $b$ ns.

\subsection*{(b) Multivariate Model (Capacitance + Fanout)}
\[
t_{\text{pd}} = a + b\,C + c\,F
\]
\begin{itemize}
  \item $a = \langle a \rangle$, $b = \langle b \rangle$, $c = \langle c \rangle$
  \item $R^2 = \langle R^2_2 \rangle$
\end{itemize}
Interpretation: Adding fanout explains $\langle more/less \rangle$ variance in incremental delay.

\subsection*{(c) Trivariate Model (Capacitance + Fanout + Input Slew) (optional)}
\[
t_{\text{pd}} = a + b\,C + c\,F + d\,S
\]
\begin{itemize}
  \item $a = \langle a \rangle$, $b = \langle b \rangle$, $c = \langle c \rangle$, $d = \langle d \rangle$
  \item $R^2 = \langle R^2_3 \rangle$
\end{itemize}
Interpretation: Higher input 
slew ($S$) values correspond to $\langle slower/faster \rangle$ transitions.  
The coefficient $d$ quantifies this dependency.

\subsection*{(d) Diagnostic Plots}
Figures~\ref{fig:cumdelay}–\ref{fig:pred} summarize timing and regression fits.  This is just a placeholder for your plots.  You obviously should also upload your Python file too.

\begin{figure}[h]
\centering
\includegraphics[width=0.9\linewidth]{delay.png}
\caption{Cumulative delay along the selected timing path.}
\label{fig:cumdelay}
\end{figure}

\begin{figure}[h]
\centering
\includegraphics[width=0.7\linewidth]{scatter_cap.png}
\caption{Incremental delay vs.\ capacitance with univariate fit line.}
\label{fig:scatter}
\end{figure}

\begin{figure}[h]
\centering
\includegraphics[width=0.7\linewidth]{predicted_vs_measured.png}
\caption{Measured vs.\ predicted incremental delay (multivariate/trivariate models).}
\label{fig:pred}
\end{figure}

\section*{4.\quad Discussion and Interpretation}

\begin{itemize}
  \item Which predictors (\texttt{Cap}, \texttt{Fanout}, \texttt{Slew}) had the largest influence?  
  \item Did adding extra variables significantly improve $R^2$ or RMSE?  
  \item Are there visible nonlinearities or outliers in residual plots?  
  \item Which stages dominate total delay? Why (large capacitance, high fanout, complex cell)?  
\end{itemize}

\section*{5.\quad Conclusions}

Summarize your findings:
\begin{itemize}
  \item Key quantitative results (coefficients, $R^2$).
  \item How well linear regression captures delay behavior.
  \item Whether nonlinear terms might help future modeling.
  \item How automated timing parsing could scale up for SoC-level analysis.
\end{itemize}

\end{document}